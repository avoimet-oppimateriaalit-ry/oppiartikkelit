% Tämä on on Oppiartikkelit-lehdessä 3.10.2014 julkaistun
% artikkelin LaTeX-lähdekoodi
% Lisenssi: http://creativecommons.org/licenses/by-sa/4.0/
% LaTeX-ohjelman voi ladata eri alustoille maksutta:
% http://www.latex-project.org/

\documentclass[twocolumn]{article}
\usepackage[utf8]{inputenc}
\title{Théveninin teoreema}
\author{Vesa Linja-aho}
\date{3.10.2014 (versio 1.0)}

\input{latex-macro.tex}
% http://code.google.com/p/latex-circuit-diagram/source/browse/latex-macro.tex
\setlength{\parskip}{1ex}
\setlength{\parindent}{0ex}
\usepackage[finnish]{babel}
\usepackage{icomma}
\usepackage{url}

\begin{document}

\maketitle

\section{Johdanto}

{\bf Portti} eli {\bf napapari} tarkoittaa kahta piirissä olevaa napaa eli sellaista solmua, johon voidaan kytkeä joku toinen piiri. Esimerkiksi auton akun navat ovat hyvä käytännön esimerkki napaparista.

{\bf Théveninin teoreeman} mukaan mikä tahansa lineaarinen piiri (ideaalista virtalähdettä lukuunottamatta\footnote{Valtonen, Lehtovuori 2011}) voidaan esittää yhdestä portista katsottuna yhden jännitelähteen ja yhden vastuksen sarjaankytkentänä.\footnote{Silvonen 2009} Tätä sarjaankytkentää kutsutaan {\bf Théveninin lähteeksi}.


\section{Théveninin lähteen muodostaminen}

Théveninin lähdejännite $E_{\rm T}$ selvitetään yksinkertaisesti laskemalla alkuperäisen piirin portin {\bf tyhjäkäyntijännite} eli jännite joka portissa on, kun siihen ei ole kytketty mitään ulkoista piiriä.


Théveninin lähteen resistanssi $R_{\rm T}$ voidaan selvittää kahdella tavalla:
\begin{description}
\item[Lähteiden sammuttamismenetelmä] Sammuttamalla kaikki piirin riippumattomat\footnote{Riippumattoman lähteen vastakohta on ohjattu lähde, jonka arvo riippuu piirin jostain toisesta jännitteestä tai virrasta, ks. luku \ref{ohjatut}.} lähteet ja laskemalla portista näkyvä resistanssi.
\item[Oikosulkuvirtamenetelmä] Selvittämällä portin {\bf oikosulkuvirta} ja soveltamalla Ohmin lakia.
\end{description}

Muodostetaan kuvan \ref{thev1} vasemman puolen piiristä Théveninin lähde (oikealla).

\begin{figure}[ht]
\begin{center}
\begin{picture}(200,55)(0,0)
\vst{0,0}{E}
\hz{0,50}{R_1}
\vz{50,0}{R_2}
\hln{0,0}{60}
\hln{50,50}{10}
\out{60,0}
\out{60,50}
\txt{100,25}{\Longleftrightarrow}
\vst{150,0}{E_{\rm T}}
\hz{150,50}{R_{\rm T}}
\hln{150,0}{50}
\out{200,0}
\out{200,50}
\end{picture}
\end{center}

\caption{Théveninin lähde.}\label{thev1}
\end{figure}

Portin jännite saadaan laskemalla vastusten läpi kulkeva virta ja kertomalla se
$R_2$:lla. Tämä portin jännite, niin sanottu {\bf tyhjäkäyntijännite}, on sama kuin Théveninin lähdejännite $E_{\rm T}$
\[
E_{\rm T}=\frac{E}{R_1+R_2}R_2
\]

$R_{\rm T}$ voidaan ratkaista kahdella tavalla.



\subsection{$R_{\rm T}$ lähteiden sammuttamismenetelmällä}
Sammutetaan piirin kaikki lähteet, minkä jälkeen lasketaan napojen välinen resistanssi.

Sammutettu jännitelähde on jännitelähde, jonka jännite on nolla volttia, eli pelkkä johdin. Sammutettu virtalähde on virtalähde, jonka virta on nolla ampeeria, eli katkaistu johdin. Eli jännitelähde sammutetaan korvaamalla se oikosululla (jolloin $U=0\V$) ja virtalähde sammutetaan korvaamalla se avoimella piirillä (jolloin $I=0\A$).

Piirissä on vain yksi lähde, jännitelähde, joka sammutetaan korvaamalla se oikosululla (kuva \ref{sammutus}).

\begin{figure}[ht]
\begin{center}
\begin{picture}(200,55)(0,0)
\vln{0,0}{50}
\hz{0,50}{R_1}
\vz{50,0}{R_2}
\hln{0,0}{60}
\hln{50,50}{10}
\out{60,0}
\out{60,50}
\txt{100,25}{\Longleftrightarrow}
\vln{150,0}{50}
\hz{150,50}{R_{\rm T}}
\hln{150,0}{50}
\out{200,0}
\out{200,50}
\end{picture}
\end{center}
\caption{Jännitelähde sammutetaan korvaamalla se johtimella.}
\label{sammutus}
\end{figure}

Nyt napojen välinen resistanssi on helppo laskea: $R_1$ ja $R_2$ ovat rinnan, joten resistanssiksi saadaan
\[
R_{\rm T}=\frac{1}{G_1+G_2}=\frac{R_1R_2}{R_1+R_2}.
\]

Tämä tapa on yleensä helpompi ja nopeampi kuin oikosulkuvirtamenetelmä ja siten suositeltava silloin kun piirissä ei ole ohjattuja lähteitä (ks. luku \ref{ohjatut}.

\subsection{$R_{\rm T}$ oikosulkuvirtamenetelmällä}

Asetetaan napojen väliin oikosulku, ja lasketaan oikosulun läpi kulkeva virta eli portin {\bf oikosulkuvirta} (Kuva \ref{oikosulkuvirta}).

\begin{figure}[ht]
\begin{center}
\begin{picture}(200,55)(0,0)
\vst{0,0}{E}
\vln{60,0}{50}
\hz{0,50}{R_1}
\vz{50,0}{R_2}
\hln{0,0}{60}
\hln{50,50}{10}
\out{60,0}
\out{60,50}
\txt{100,25}{\Longleftrightarrow}
\vst{150,0}{E_{\rm T}}
\vln{200,0}{50}
\hz{150,50}{R_{\rm T}}
\hln{150,0}{50}
\out{200,0}
\out{200,50}
\di{60,25}{I_{\rm K}}
\di{200,25}{I_{\rm K}}
\end{picture}
\end{center}
\caption{Oikosulkuvirran määrääminen}
\label{oikosulkuvirta}
\end{figure}

Koska vastus $R_2$ on ohitettu oikosululla, sen läpi ei kulje virtaa ja oikosulkuvirran suuruus on
\[
I_{\rm K}=\frac{E}{R_1}
\]
ja vastuksen $R_{\rm T}$ arvoksi saadaan (soveltamalla Ohmin lakia oikeanpuoleiseen kuvaan eli Théveninin lähteeseen)
\[
R_{\rm T}=\frac{E_{\rm T}}{I_{\rm K}}=\frac{E_{\rm T}}{\frac{E}{R_1}}=\frac{\frac{E}{R_1+R_2}R_2}{\frac{E}{R_1}}=
\frac{R_1R_2}{R_1+R_2}.
\]
Tulos on sama kuin lähteiden sammuttamismenetelmälläkin. 

\section{Lopputuloksen tarkastelua}

Théveninin lähde käyttäytyy ulkopuolelta katsottuna samalla tavalla kuin alkuperäinen muunnettu piiri. Esimerkiksi jos komponenttiarvot ovat
\[
R_1=1\kohm \quad R_2=1\kohm \quad E_{\rm T}=12 \V
\]

niin kuvan \ref{thevlopullinen} vasemman- ja oikeanpuoleisia piirejä ei pysty mitenkään erottamaan toisistaan napojen välistä tehtävillä mittauksilla.

\begin{figure}[ht]
\begin{center}
\begin{picture}(200,55)(0,0)
\vst{0,0}{E}
\hz{0,50}{1\kohm}
\vz{50,0}{1\kohm}
\hln{0,0}{60}
\hln{50,50}{10}
\out{60,0}
\out{60,50}
\txt{100,25}{\Longleftrightarrow}
\vst{150,0}{6\V}
\hz{150,50}{500\ohm}
\hln{150,0}{50}
\out{200,0}
\out{200,50}
\end{picture}
\end{center}

\caption{Théveninin lähde.}\label{thevlopullinen}
\end{figure}

Molemmissa piireissä portissa on $6\V$ tyhjäkäyntijännite, oikosulkuvirta on $12\mA$ ja jos porttiin kytkee minkä suuruisen vastuksen (tai muun komponentin) tahansa, on vastuksen virta ja jännite sama molemmissa piireissä. Jos piirit olisi rakennettu kahden mustan laatikon sisään, ulkopuolisilla mittauksilla ei voi mitenkään selvittää, kumpi piiri on missäkin laatikossa.\footnote{Paitsi herkän lämpökameran avulla: vasemmanpuoleisessa piirissä kulkee jatkuvasti $6\mA$ virta vastusten $R_1$ ja $R_2$ läpi, joten piiri lämpenee kuormittamattomanakin $6\mA\cdot12\V=72\rm \, mW$ teholla, toisin kuin oikeanpuoleinen piiri.}

\section{Toinen esimerkki}

Muodostetaan alla olevasta piiristä Théveninin lähde. Kaikki komponenttiarvot ovat ykkösiä.
(Vastukset ovat jokainen $1 \ohm$ ja virtalähde $J=1\A$.)

\begin{center}
\begin{picture}(150,50)(0,0)
\vj{0,0}{J}
\vz{50,0}{R_1}
\vz{100,0}{R_3}
\hz{50,50}{R_2}
\out{150,0}
\out{150,50}
\hln{0,0}{150}
\hln{100,0}{50}
\hln{0,50}{50}
\hln{100,50}{50}
\end{picture}
\end{center}

{\bf Ratkaisu:} Selvitetään ensin Théveninin jännite $E_{\rm T}$. Tämän voi tehdä esimerkiksi lähdemuunnoksen ja jännitteenjakosäännön avulla:
\begin{center}
\begin{picture}(150,50)(35,0)
\vst{0,0}{J_1R_1\hspace{-1.7cm}}
\hz{0,50}{R_1}
\vz{100,0}{R_3}
\hz{50,50}{R_2}
\out{150,0}
\out{150,50}
\hln{0,0}{150}
\hln{100,0}{50}
\hln{100,50}{50}
\du{110,0}{E_{\rm T}=\frac{J_1R_1}{R_1+R_2+R_3}R_3=\frac{1}{3}\V}
\end{picture}
\end{center}

Ratkaistaan seuraavaksi Théveninin lähteen resistanssi $R_{\rm T}$. Helpoiten tämä onnistuu sammuttamalla
lähteet ja laskemalla portista näkyvä resistanssi (toinen tapa olisi oikosulkuvirran selvittäminen).

Resistanssin
voi laskea joko alkuperäisestä tai muunnetusta piiristä, lopputulos on sama koska lähdemuunnos ei muuta piiri ulkoista toimintaa. Lasketaan muunnetusta piiristä, eli sammutetaan jännitelähde:
\begin{center}
\begin{picture}(150,50)(35,0)
\vln{0,0}{50}
\hz{0,50}{R_1}
\vz{100,0}{R_3}
\hz{50,50}{R_2}
\out{150,0}
\out{150,50}
\hln{0,0}{150}
\hln{100,0}{50}
\hln{100,50}{50}
\txt{170,25}{R_{\rm T}=\frac{1}{\frac{1}{R_1+R_2}+\frac{1}{R_3}}=\frac{2}{3}\ohm}
\end{picture}
\end{center}
Vastukset $R_1$ ja $R_2$ ovat sarjassa, ja tämä sarjaankytkentä on rinnan $R_3$:n kanssa.
Nyt $E_{\rm T}$ ja $R_{\rm T}$ tiedetään, joten meillä on valmis Théveninin lähde:

\begin{center}
\begin{picture}(50,50)(0,0)
\vst{0,0}{E_{\rm T}=\frac{1}{3}\V}
\hln{0,0}{50}
\hz{0,50}{R_{\rm T}=\frac{2}{3}\ohm}
\out{50,50}
\out{50,0}
\end{picture}
\end{center}

\section{Harjoitustehtäviä}

Muodosta Théveninin lähde alla olevista piireistä. Kaikki komponenttiarvot = 1.
\begin{center}
\begin{picture}(150,50)(0,0)
\vj{0,0}{J_1}
\vz{50,0}{R_1}
\vz{100,0}{R_2}
\hst{50,50}{E}
\out{150,0}
\out{150,50}
\hln{0,0}{150}
\hln{100,0}{50}
\hln{0,50}{50}
\hln{100,50}{50}
\end{picture}
\end{center}

\begin{center}
\begin{picture}(150,50)(0,0)
\vst{0,0}{E}
\vz{50,0}{R_2}
\vj{100,0}{J}
\hz{50,50}{R_3}
\out{150,0}
\out{150,50}
\hln{0,0}{150}
\hln{100,0}{50}
\hz{0,50}{R_1}
\hln{100,50}{50}
\end{picture}
\end{center}
{\tiny Ratkaisu: $E_{\rm T}=1\V$ $R_{\rm T}=0,5\ohm$ ja $E_{\rm T}=2\V$ $R_{\rm T}=1,5\ohm$}




\section{Sovelluksia}

Théveninin lähde helpottaa joitain virtapiirilaskuja, kun monimutkainen osa piiristä voidaan pelkistää jännitelähteeksi ja vastukseksi.

Moni käytännön virtapiiri voidaan ajatella Théveninin lähteenä: esimerkiksi pistorasiasta voidaan mitata tyhjäkäyntijännite yleismittarilla ja sisäinen resistanssi asennustesterillä, jolloin saadaan laskettua oikosulkuvirta, jonka tulee olla riittävä johdonsuojalaitteiden toiminnalle.

Käytännön elektroniikkalaitteet suunnitellaan yleensä tietyn toiminnon toteuttavina lohkoina ja on suunnittelijan kannalta kätevää, jos yhden lohkon toiminta voidaan mallintaa yksinkertaisella sijaiskytkennällä. Esimerkiksi vahvistimen lähtöä voidaan mallintaa Théveninin lähteellä, jolla on tietty lähdejännite ja lähtöresistanssi.

\section{Ohjatut lähteet}\label{ohjatut}

Lähteiden sammuttamismenetelmässä sammutetaan kaikki riippumattomat lähteet ja lasketaan portista näkyvä resistanssi.

Resistanssin laskeminen piiristä, josta kaikki lähteet on sammutettu, on suoraviivaista, joten lähteiden sammuttamismenetelmä on kätevä silloin kun kaikki lähteet piirissä ovat riippumattomia.

Riippumattoman lähteen vastakohta on {\bf ohjattu lähde}, jonka arvo riippuu piirin jostain toisesta jännitteestä tai virrasta.

{\bf Ohjattuja lähteitä puolestaan ei saa sammuttaa} portista näkyvää resistanssia laskettaessa, ja resistanssin laskeminen piiristä jossa on ohjattuja lähteitä on tavallisesti työlästä.

Mikäli piirissä on ohjattuja lähteitä, Théveninin resistanssia selvitettäessä kannattaakin käyttää oikosulkuvirtamenetelmää.

Oikosulkuvirtamenetelmä toimii aina, olipa piirissä ohjattuja lähteitä tai ei.

\section{Muuta}

Théveninin lähde voidaan muodostaa myös pelkistämällä piiriä piirimuunnoksien avulla kunnes lopputuloksena on jännitelähteen ja vastuksen sarjaankytkentä.

Théveninin teoreema pätee myös vaihtosähköpiireille: tällöin lähteenä on vaihtojännitelähde ja resistanssin tilalla on impedanssi.

Théveninin teoreeman kehitti ensimmäisenä vuonna 1853 saksalainen fyysikko Hermann von Helmholtz. Vuonna 1883 ranskalainen lennätininsinööri Léon Charles Thévenin päätyi itsenäisesti samaan tulokseen.\footnote{Silvonen 2009}

Théveninin teoreemalle läheistä sukua on {\bf Nortonin teoreema}, jossa piiri esitetään virtalähteen ja resistanssin rinnankytkentänä, {\bf Nortonin lähteenä}. Nortonin lähde on käytännössä Théveninin lähde, jolle on tehty jännitelähde--virtalähdemuunnos.


\section{Lähteet}

Kimmo Silvonen: {\em Sähkötekniikka ja piiriteoria}. Otatieto, Helsinki. 2009.

Martti Valtonen, Anu Lehtovuori: {\em Piirianalyysi. Osa 1, Tasa- ja vaihtovirtapiirien analyysi}. Unigrafia, Helsinki. 2011.

\section*{Kirjoittajasta}

Artikkelin kirjoittaja Vesa Linja-aho, on koulutukseltaan sähkötekniikan ja elektroniikan diplomi-insinööri. Linja-aho työskentelee autoelektroniikan lehtorina Metropolia-ammattikorkeakoulussa. Aikaisemmin hän on toiminut yliopisto-opettajana Teknillisen korkeakoulun (nyk. Aalto-yliopisto) Teoreettisen sähkötekniikan laboratoriossa.

\section*{Oppiartikkelit-sarja}

Tämä artikkeli on julkaistu Oppiartikkelit-lehdessä 3.10.2014. Oppiartikkelit ovat lyhyitä, vakiintuneeseen tieteelliseen tietoon perustuvia artikkeleita, jotka toinen alan asiantuntija on vertaisarvioinut.

Oppiartikkelit-lehti on vapaasti luettavissa osoitteessa \url{http://oppiartikkelit.fi}

{\bf Asiasanat}: Théveninin teoreema, Théveninin lähde, oikosulkuvirtamenetelmä, lähteiden sammuttamismenetelmä, tyhjäkäyntijännite, oikosulkuvirta, Nortonin lähde

{\bf Esitietosuositukset}: Kirchhoffin jännitelaki, Kirchhoffin virtalaki, Ohmin laki, lähdemuunnos, jännitelähde-virtalähdemuunnos



\end{document}
